\documentclass[11pt]{article}
\usepackage{setspace}
\usepackage{geometry}
\usepackage{xcolor}
\usepackage{natbib}
\usepackage{xr}

\externaldocument{gc-transfer}

\geometry{margin=1in}

\begin{document}

\begin{center}
{\Large \textbf{Responses to Reviewer 5}} \\[6pt]
%Manuscript ID: \textbf{diagnostics-3950051} \\
%Title: \textit{Cross-Cancer Transfer Learning for Gastric Cancer Risk Prediction from Electronic Health Records}
\end{center}

\vspace{1em}

%%%%%%%%%%%%%%%%%%%%%%%%%%%%%%%%%%%%%%%%%%%%%%%%%%%%%%%%%%%%%%
%\section*{Summary}
We sincerely thank the reviewers for their careful reading of our manuscript and for providing insightful comments that helped us improve the clarity and rigor of the work. Below we provide detailed, point-by-point responses to all comments. All revisions are highlighted in the revised manuscript in \textcolor{red}{red}, and the locations of the changes are indicated in each response.

%%%%%%%%%%%%%%%%%%%%%%%%%%%%%%%%%%%%%%%%%%%%%%%%%%%%%%%%%%%%%%
%\section*{}

\subsection*{Comment 1}
%\textit{
congratulations It is a great idea, and effort.  but I am not sure about the impact and interest for readers this subject.
%}

\subsection*{Response 1}
We thank the reviewer for the positive feedback on the idea and effort, and
we appreciate the opportunity to clarify the clinical impact
of our work.
Our main goal is to provide a framework for building gastric cancer
(GC) risk models under label scarcity by reusing routinely collected structured
EHR data and related non-GC cancer cohorts that many institutions already have.

To better highlight this contribution, we have revised the manuscript in several
places.
First, in the Introduction (p.~X, paragraph~Y, lines~A--B), we now more clearly
explain that the proposed cross-cancer transfer framework is directly motivated
by health systems that see relatively few GC cases but do manage other GI/HPB
malignancies, and that the approach does not require imaging or free-text notes,
making it easier to deploy as a background risk score in routine workflows.
\begin{quote}
From a clinical perspective, the proposed framework is especially
relevant for health systems that see relatively few GC cases but routinely manage
other gastrointestinal and hepatopancreatobiliary malignancies.
In such settings, our study shows how existing
structured EHR data and non-GC cancer cohorts can be reused to build a risk model
for GC under label scarcity without requiring new data collection.
Beyond GC, the same cross-cancer transfer framework can be directly adapted to
other underrepresented cancers or rare tumor subtypes where labeled data are
limited but related malignancies are more prevalent.
\end{quote}
Second, in the Discussion (p.~X, paragraph~Y, lines~C--D), we have modified a
paragraph on clinical implications and generalizability, emphasizing that:
(i) the model is intended to support triage and prioritization for earlier
endoscopic evaluation rather than replace diagnostic procedures; and
(ii) the same design can be adapted to other underrepresented cancers or rare
tumor subtypes where labeled data are limited but related source cancers are
more common.
\begin{quote}
From a clinical standpoint, the proposed framework is not meant to replace
............
\end{quote}
Finally, we strengthened the Abstract conclusion (p.~1, lines~E--F) to explicitly
state the potential utility for institutions with limited GC labels but existing
GI/HPB cohorts.
\begin{quote}
Cross-cancer transfer on EHRs suggests a sample-efficient, deployment-ready strategy for GC risk modeling, particularly for health systems with limited GC labels but existing GI/HPB cancer cohorts, and may help prioritize endoscopic evaluation and follow-up within routine workflows. By reusing related cancer cohorts to bootstrap GC representation learning, the approach strengthens early-risk identification under label scarcity and supports clinician decision-making at the point of care.
\end{quote}

%We hope that these revisions better communicate why this cross-cancer transfer
%framework is relevant and of interest to both clinical and methodological
%audiences.



%%%%%%%%%%%%%%%%%%%%%%%%%%%%%%%%%%%%%%%%%%%%%%%%%%%%%%%%%%%%%%
\section*{Response to Comments on English Language}

\subsection*{Point 1}
\textit{[Copy Reviewer comment on English language, if present.]}

\subsection*{Response}
\textcolor{red}{We have carefully revised the manuscript for clarity, grammar, and stylistic quality.}

%%%%%%%%%%%%%%%%%%%%%%%%%%%%%%%%%%%%%%%%%%%%%%%%%%%%%%%%%%%%%%
\section*{Additional Clarifications to the Editor}
We respectfully note that the revisions requested by the reviewers have been fully incorporated into the manuscript. All modifications are highlighted in \textcolor{red}{red}. We appreciate the editor’s guidance and remain available to provide any additional information.

%%%%%%%%%%%%%%%%%%%%%%%%%%%%%%%%%%%%%%%%%%%%%%%%%%%%%%%%%%%%%%
\bibliographystyle{plainnat}
\bibliography{references} % Replace ‘references.bib’ with your bib file name.

\end{document}
